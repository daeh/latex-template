\section{Introduction}

My research highlights the critical role of context, and of prediction, in emotion understanding. Contextual information shapes forward predictions of how others will interpret external events in relation to their mental contents (beliefs, desires, moral values, costs, etc.). These forward predictions in turn constrain ill-posed inverse inferences, allowing people to explain others' expressions and behavior \citep{saxe2017cop}. My results support the view that forward predictions guide inverse inferences about expressions. Contextual cues about what did happen, what could have happened, what someone wanted to happen, and what someone believed would happen, constrain the predictions generated by people's intuitive theory \citep{houlihan2018cogsci}. The contextual shaping of forward predictions in turn constrains the emotions attributed to expressions \citep{anzellotti2021emotion}, and the prior experiences inferred from from expressions \citep{houlihan2022emotionreasoning}. Knowing specifically what situations people were reacting to changes the emotions attributed to their expressions \citep{anzellotti2021emotion, houlihan2022thesis}. Knowing merely what events \emph{might} have transpired, but not what actually happened, also changes emotion attribution \citep{houlihan2022emotionreasoning, anzellotti2021emotion}.

